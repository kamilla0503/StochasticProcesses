\documentclass[11pt]{article}
\usepackage{amsmath, amssymb, amscd, amsthm, amsfonts}
\usepackage{graphicx}
\usepackage{hyperref}

\usepackage{tikz}
\usetikzlibrary{arrows}

\oddsidemargin 0pt
\evensidemargin 0pt
\marginparwidth 40pt
\marginparsep 10pt
\topmargin -20pt
\headsep 10pt
\textheight 8.7in
\textwidth 6.65in
\linespread{1.2}

\title{ }
\author{ }
\date{}

\newtheorem{theorem}{Theorem}
\newtheorem{lemma}[theorem]{Lemma}
\newtheorem{conjecture}[theorem]{Conjecture}

\newcommand{\rr}{\mathbb{R}}

\newcommand{\al}{\alpha}
\DeclareMathOperator{\conv}{conv}
\DeclareMathOperator{\aff}{aff}

\begin{document}

\maketitle

\begin{abstract}
 
\end{abstract}

\section{ }\label{section:introduction}

Suppose, that we have the stochastic process which consists of $N$ internal states. In each state, the process has waiting time distributed exponentially . For each state parameters $\tau_i \sim Exp(\lambda_i)$ are independent. 
For simplicity, we assume $N$ as a fixed parameter. 
\medskip
%\begin{figure}
%\centerline{\includegraphics[scale=1]{fig1-Tverberg}}
%\caption{An example of a Tverberg partition.  The partition is not unique.}
%\end{figure}
 
 \tikzstyle{int}=[draw, fill=blue!20, minimum size=2em]
 \tikzstyle{init} = [pin edge={to-,thin,black}]
 \begin{tikzpicture}[node distance=2.5cm,auto,>=latex']
 \node [int ] (a) {$\lambda_1$};
 \node (b) [left of=a,node distance=2cm, coordinate] {a};
 \node [int ] (c) [right of=a] {$\lambda_2$};
 \node [int ] (d) [right of=c] {$\lambda_i$};
 \node [int ] (e) [right of=d] {$\lambda_N$};
 \node [coordinate] (end) [right of=e, node distance=2cm]{};
 \path[->] (b) edge node {  } (a);
 \path[->] (a) edge node { } (c);
 \path[->] (c) edge node {...} (d);
 \path[->] (d) edge node {...} (e);
 \draw[->] (e) edge node { } (end) ;
 \end{tikzpicture}
 \medskip
  \par Data observations $X_j$, $j=1..m$ is a sum of waiting times: $X = \sum_i^N \tau_i$. Our aim is to estimate parameters $\lambda_i$.
  

\subsection{ }
 

 

\bibliographystyle{alpha}
\bibliography{references} % see references.bib for bibliography management

\end{document}
